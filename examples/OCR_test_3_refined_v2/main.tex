\documentclass[12pt,a4paper]{article}
\usepackage{amsmath,amssymb,amsfonts}
\usepackage{graphicx}
\usepackage{geometry}
\geometry{margin=2.5cm}
\setlength{\parindent}{0pt}
\setlength{\parskip}{6pt}


\begin{document}


% Page 1
\section*{HeySeen Benchmark: Geometric \& Structural Stress Test}

\section{003 HeySeen Laboratory - Stress Scenario}

\section{2026 Ngày 4 tháng 2 năm}

\begin{figure}[h]
  \centering
  \includegraphics[width=0.8\textwidth]{examples/OCR_test_3_refined_v2/images/page_000_table_03.png}
  \caption{Figure from page 1}
\end{figure}

$P(r,\theta)$

x

1

Hệ tọa độ cực liên hệ với tọa độ Descartes qua:


\[ \begin{cases} x = r \cos \theta \\ y = r \sin \theta \end{cases} \implies x^2 + y^2 = r^2 \]


\begin{figure}[h]
  \centering
  \includegraphics[width=0.8\textwidth]{examples/OCR_test_3_refined_v2/images/page_000_table_12.png}
  \caption{Figure from page 1}
\end{figure}


\[ \mathbf{M} = \begin{bmatrix} 1 & 0 & a \\ 0 & 1 & b \\ \hline 0 & 0 & 1 \end{bmatrix} \otimes \begin{pmatrix} \lambda_1 & \mathbf{0} \\ & \ddots & \\ \mathbf{0} & & \lambda_n \end{pmatrix} \]


Tính chất của định thức ma trận khối:


\[ \det\begin{pmatrix} A & B \\ 0 & D \end{pmatrix} = \det(A)\det(D) \]


(1)


\newpage

% Page 2
Test 3: Hóa học và Chỉ số phức tạp Texify đôi khi nhầm lẫn giữa công thức toán và công thức hóa học. Hãy xem HeySeen xử lý gói 'mhchem' như thế nào.

3.1 Phản ứng nhiệt nhôm

$2 \text{Al} + \text{Fe}_2\text{O}_3 \xrightarrow{\text{t}^o} \text{Al}_2\text{O}_3 + 2 \text{Fe}$

\begin{figure}[h]
  \centering
  \includegraphics[width=0.8\textwidth]{examples/OCR_test_3_refined_v2/images/page_000_table_07.png}
  \caption{Figure from page 2}
\end{figure}

\begin{figure}[h]
  \centering
  \includegraphics[width=0.8\textwidth]{examples/OCR_test_3_refined_v2/images/page_000_table_11.png}
  \caption{Figure from page 2}
\end{figure}

\begin{figure}[h]
  \centering
  \includegraphics[width=0.8\textwidth]{examples/OCR_test_3_refined_v2/images/page_000_table_15.png}
  \caption{Figure from page 2}
\end{figure}


\[ \int_{i=1}^{n} \frac{1}{i} \]

\[ \int \sin(x) dx \]


\begin{figure}[h]
  \centering
  \includegraphics[width=0.8\textwidth]{examples/OCR_test_3_refined_v2/images/page_000_table_29.png}
  \caption{Figure from page 2}
\end{figure}


\[ E[\omega] = \sum_{j=1}^{m} p_j \cdot u(\omega_j) \]


\begin{itemize}
  \item ước 3: Nếu $E[\omega] > \tau$, thực hiện phép gán $x \leftarrow x + \Delta x$.
\end{itemize}

Trạng thái hệ thống được mô tả bởi toán tử Hamiltonian:

$\hat{H}\Psi = E\Psi$ với $\hat{H} = -\frac{\hbar^2}{2m}\nabla^2 + V(\mathbf{r})$


\newpage

\end{document}
